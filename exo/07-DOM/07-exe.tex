\documentclass[10pt,a4paper]{article}
\usepackage[utf8]{inputenc}
\usepackage[french]{babel}
\usepackage[T1]{fontenc}
\usepackage{amsmath}
\usepackage{amsfonts}
\usepackage{amssymb}
\usepackage[left=2cm,right=2cm,top=2cm,bottom=2cm]{geometry}
\usepackage{hyperref}
\usepackage{graphicx}

\usepackage{fancyhdr}

\pagestyle{fancy}
\lhead{IFNTI Sokodé -- L3}
\rhead{Javascript - Document Object Model}
\cfoot{Institut de Formation aux Normes et Technologies de l'Informatique\\300 BP 40, Sokodé TOGO -- Tél. : +228 90 91 81 41}
\rfoot{\thepage}

\title{JavaScript\\DOM Exercice\footnote{Fortement inspiré de \url{https://openclassrooms.com/fr/courses/5543061-ecrivez-du-javascript-pour-le-web/}}}
\author{IFNTI Sokodé -- L3}
\date{28 Avril 2021}

%https://www.pierre-giraud.com/javascript-apprendre-coder-cours/
%https://openclassrooms.com/fr/courses/6175841-apprenez-a-programmer-avec-javascript/6278392-declarez-des-variables-et-modifiez-leurs-valeurs
\begin{document}
Récupérer le code sur Moodle.

Modifier le JavaScript pour récupérer :
\begin{itemize}
	\item l'élément ayant pour ID ``main-content" grâce à son ID
	\item l'élément ayant pour classe ``important" 
	\item les éléments ayant pour type ``article"
	\item du premier élément :
	\begin{itemize}
		\item de type li
		\item qui sont dans une liste ul 
		\begin{itemize}
			\item ayant la classe ``important"
			\item qui est incluse dans un ``article"
		\end{itemize}				
	\end{itemize}
	\item à partir de la réponse précédente : récupérez l'élément li suivant de la liste ul
\end{itemize}
\end{document}